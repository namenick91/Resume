% template author: LianTze Lim (liantze@gmail.com)

\documentclass[a4paper,skipsamekey,11pt,english]{curve}

% Uncomment to enable Chinese; needs XeLaTeX
% \usepackage{ctex}

% Default biblatex style used for the publication list is APA6. If you wish to use a different style or pass other options to biblatex you can change them here. 
\PassOptionsToPackage{style=ieee,sorting=ydnt,uniquename=init,defernumbers=true}{biblatex}

% Most commands and style definitions are in settings.sty.
\usepackage{settings}

% If you need to further customise your biblatex setup e.g. with \DeclareFieldFormat etc please add them here AFTER loading settings.sty. For example, to remove the default "[Online] Available:" prefix before URLs when using the IEEE style:
\DefineBibliographyStrings{english}{url={\textsc{url}}}

%% Only needed if you want a Publication List
\addbibresource{own-bib.bib}

%% Specify your last name(s) and first name(s) (as given in the .bib) to automatically bold your own name in the publications list. 
%% One caveat: You need to write \bibnamedelima where there's a space in your name for this to work properly; or write \bibnamedelimi if you use initials in the .bib
% \mynames{Lim/Lian\bibnamedelima Tze}

%% You can specify multiple names like this, especially if you have changed your name or if you need to highlight multiple authors. See items 6–9 in the example "Journal Articles" output.
\mynames{Lim/Lian\bibnamedelima Tze,
  Wong/Lian\bibnamedelima Tze,
  Lim/Tracy,
  Lim/L.\bibnamedelimi T.}
%% MAKE SURE THERE IS NO SPACE AFTER THE FINAL NAME IN YOUR \mynames LIST

% Change the fonts if you want
\ifxetexorluatex % If you're using XeLaTeX or LuaLaTeX
  \usepackage{fontspec} 
  % \usepackage[p,osf,swashQ]{cochineal}
  % \usepackage[medium,bold]{cabin}
  % \usepackage[varqu,varl,scale=0.9]{zi4}

  \setmainfont{Open Sans} % Lato, Open Sans
  \setsansfont{Verdana}
\else % If you're using pdfLaTeX or latex
  \usepackage[T1]{fontenc}
  \usepackage[p,osf,swashQ]{cochineal}
  \usepackage{cabin}
  \usepackage[varqu,varl,scale=0.9]{zi4}
\fi

% Change the page margins if you want
% \geometry{left=1cm,right=1cm,top=1.5cm,bottom=1.5cm}

% Change the colours if you want
% \definecolor{SwishLineColour}{HTML}{606c38}
\definecolor{MarkerColour}{HTML}{283618}

% Change the item prefix marker if you want
% \prefixmarker{$\diamond$}

%% Photo is only shown if "fullonly" is included
% \includecomment{fullonly}
\excludecomment{fullonly}


%%%%%%%%%%%%%%%%%%%%%%%%%%%%%%%%%%%%%%


\leftheader{%
  {\LARGE\bfseries\sffamily Никита Очкин}

  \makefield{\faEnvelope[regular]}{\href{mailto:nkt.ochkin@gmail.com}{\texttt{nkt.ochkin@gmail.com}}}
  \makefield{\faGithub}{\href{https://github.com/namenick91}{\texttt{namenick91}}}
  \makefield{\faPaperPlane}{\href{https://t.me/namenick9/}{\texttt{namenick9}}}

  %% Next line
  % \makefield{\faGlobe}{\url{http://example.example.org/}}
  % You can use a tabular here if you want to line up the fields.
}

\rightheader{~}
\begin{fullonly}
\photo[r]{photo}
\photoscale{0.13}
\end{fullonly}

\title{Curriculum Vitae}

\begin{document}
\makeheaders[c]

% \makerubric{employment}
\makerubric{education}

% If you're not a researcher nor an academic, you probably don't have any publications; delete this line.
%% Sometimes when a section can't be nicely modelled with the \entry[]... mechanism; hack our own and use \input NOT \makerubric
\input{publications}

% \begin{rubric}{Проектная деятельность}

\entry*%
    % \textbf{Forecasting Transformers (Informer, Autoformer, Performer)} -
    % долгосрочный прогноз мультивариантных рядов; собрал воспроизводимый
    % PyTorch-pipeline, интегрировал компоненты из SOTA-моделей в единую архитектуру,
    % провел абляции модулей ConvStem/FAVOR+/SeriesDecomposition (вклад: ΔMAE/ΔRMSE);
    % \textbf{MAE/MSE/RMSE: [0.471/0.447/0.668]} (avg@24--720),
    % vs baseline: \textbf{$-$[0.259/0.452/0.266Δ]} (ETTh1 датасет).

    \textbf{Forecasting Transformers (Informer, Autoformer, Performer)} - 
    долгосрочный прогноз мультивариантных рядов; собрал воспроизводимый PyTorch-pipeline, 
    интегрировал компоненты из SOTA-моделей в единую архитектуру, провёл абляции модулей 
    ConvStem/FAVOR+/SeriesDecomposition. 

    \textbf{ETTh1 датасет; avg@24--720: MAE/MSE/RMSE: 0.471/0.447/0.668}. 

    \textbf{Улучшение относительно baseline (Informer): 
    $\Delta$0.259/$\Delta$0.452/$\Delta$0.265}.


\entry*%
	\textbf{Simpsons Character Classification (EfficientNetV2-S)} - 
    multi-class классификация изображений, двухфазный fine-tuning, 
    RandAugment/CutMix/MixUp, AdamW+Cosine, early stopping; 
    
    \textbf{Micro-F1: 0.99574}.

\entry*%
	\textbf{Semantic Segmentation (U-Net/SegNet)} - 
    бинарная сегментация мед. изображений; BCE/Dice/Focal, 
    трекинг IoU, Early Stopping; 
    
    \textbf{IoU(SegNet+Dice): 0.804}.

\entry*%
	\textbf{Customer Churn (CatBoost)} - 
    бинарная классификация оттока клиентов; очистка/типизация, фичеинжиниринг, 
    k-fold, Optuna; 
    
    \textbf{ROC-AUC: 0.859}.

\entry*%
	\textbf{Game of Thrones Survival (RandomForest)} - 
    бинарная классификация; препроцессинг, построение кастомного sklearn-пайплайна,
    тьюнинг гиперпараметров; RF после бенчмарка моделей; 
    
    \textbf{ROC-AUC: 0.7953, Accuracy: 0.8045}.
\end{rubric}

\makerubric{petprojects}
\makerubric{skills}
% \makerubric{misc}

% \makerubric{referee}
% \input{referee-full}

\end{document}
